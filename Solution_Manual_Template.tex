%%%%%%%%%%%%%%%%%%%%%%%%%%%%%%%%%%%%%%%%%%%%%%%%%%%%%%%%%%%%%%%%%%%%%%%%%%%%%%%%
%                     PROFESSIONAL SOLUTION MANUAL TEMPLATE                   %
%                            Enhanced Professional Version                     %
%                                                                              %
% USAGE INSTRUCTIONS:                                                          %
% 1. Replace "Your Name" with your actual name                                %
% 2. Replace "Course Title or Subject" with your course/subject               %
% 3. Replace "your.email@example.com" with your email                         %
% 4. Modify chapter and section titles as needed                              %
% 5. Replace placeholder problems and solutions with actual content           %
% 6. Customize colors in the color definitions section if desired             %
%                                                                              %
% FEATURES:                                                                    %
% - Professional styling with tcolorbox environments                          %
% - Problem/Solution environments with automatic numbering                    %
% - Key Insight and Key Observation environments                              %
% - Standard theorem environments (Definition, Lemma, Theorem, etc.)         %
% - Cross-referencing with cleveref                                          %
% - Hyperlinked table of contents and references                             %
% - Professional color scheme                                                 %
% - Index and bibliography support                                           %
%                                                                              %
% COMPILE INSTRUCTIONS:                                                        %
% Run: pdflatex -> pdflatex -> makeindex -> pdflatex                         %
% (Multiple runs needed for cross-references and index)                      %
%%%%%%%%%%%%%%%%%%%%%%%%%%%%%%%%%%%%%%%%%%%%%%%%%%%%%%%%%%%%%%%%%%%%%%%%%%%%%%%%
\documentclass[11pt, oneside, openany]{book}

%------------------------------------------------------------------------------
% 1. ENCODING, LANGUAGE, GEOMETRY
%------------------------------------------------------------------------------
\usepackage[utf8]{inputenc}
\usepackage[T1]{fontenc}
\usepackage[english]{babel}
\usepackage{geometry}
\geometry{
    a4paper, 
    margin=1.1in, 
    headheight=15pt,
    footskip=30pt,
    marginparwidth=2cm
}

%------------------------------------------------------------------------------
% 2. MATH PACKAGES
%------------------------------------------------------------------------------
\usepackage{amsthm, amssymb, mathtools}

%------------------------------------------------------------------------------
% 3. FONTS & TYPOGRAPHY
%------------------------------------------------------------------------------
\usepackage{lmodern} % Latin Modern fonts - widely available
\usepackage{microtype} % Improved typography
\usepackage{setspace} % Line spacing control
\setstretch{1.1} % Slightly increased line spacing for readability

% Better section formatting
\usepackage{titlesec}
\titleformat{\chapter}[display]
  {\normalfont\huge\bfseries\color{NavyBlue}}
  {\chaptertitlename\ \thechapter}{20pt}{\Huge\color{black}}
\titleformat{\section}
  {\normalfont\Large\bfseries\color{NavyBlue}}
  {\thesection}{1em}{}
\titleformat{\subsection}
  {\normalfont\large\bfseries\color{DarkSlateGray}}
  {\thesubsection}{1em}{}

%------------------------------------------------------------------------------
% 4. COLOR, BOXES, ENUMERATION
%------------------------------------------------------------------------------
\usepackage[dvipsnames, svgnames, table]{xcolor}
\usepackage{enumitem}
\usepackage[many]{tcolorbox}
\tcbuselibrary{theorems, breakable, skins}

% Enhanced color palette
\definecolor{primaryblue}{RGB}{41, 98, 255}
\definecolor{secondaryblue}{RGB}{66, 133, 244}
\definecolor{lightblue}{RGB}{232, 240, 254}
\definecolor{darkgray}{RGB}{55, 71, 79}
\definecolor{lightgray}{RGB}{245, 246, 247}
\definecolor{accentgreen}{RGB}{52, 168, 83}
\definecolor{warningorange}{RGB}{255, 171, 0}

% Problem and solution colors
\definecolor{problemback}{RGB}{232, 240, 254}
\definecolor{problembar}{RGB}{41, 98, 255}
\definecolor{solutionback}{RGB}{249, 250, 251}

% Definition and theorem colors
\definecolor{definitionback}{RGB}{240, 253, 244}
\definecolor{definitionbar}{RGB}{52, 168, 83}
\definecolor{lemmaback}{RGB}{255, 248, 225}
\definecolor{lemmabar}{RGB}{255, 171, 0}
\definecolor{insightback}{RGB}{255, 245, 238}
\definecolor{insightbar}{RGB}{255, 87, 34}

%------------------------------------------------------------------------------
% 5. TOC & INDEXING
%------------------------------------------------------------------------------
\usepackage{tocloft}
\cftsetindents{section}{0em}{2.5em}
\cftsetindents{subsection}{2.5em}{3em}
\setlength{\cftbeforechapskip}{1em}
\setlength{\cftbeforesecskip}{0.5em}
\usepackage{makeidx}
\makeindex

%------------------------------------------------------------------------------
% 6. HYPERLINKING & REFERENCING
%------------------------------------------------------------------------------
\usepackage{hyperref}
\hypersetup{
    pdftitle={Solution Manual},
    pdfauthor={Your Name},
    colorlinks=true,
    linkcolor=NavyBlue,
    citecolor=ForestGreen,
    urlcolor=DarkOrchid,
    bookmarksopen=true,
    bookmarksnumbered=true,
    pdfdisplaydoctitle=true
}
% cleveref should be loaded after hyperref
\usepackage[capitalize, nameinlink]{cleveref}

%------------------------------------------------------------------------------
% 7. THEOREM STYLES & CUSTOM ENVIRONMENTS
%------------------------------------------------------------------------------
\theoremstyle{definition}
\newtheorem{lemma}{Lemma}[section]
\newtheorem{theorem}{Theorem}[section]
\newtheorem{corollary}{Corollary}[section]
\newtheorem{definition}{Definition}[section]
\newtheorem{remark}{Remark}[section]

%%% ENHANCED PROBLEM ENVIRONMENT
\newtcbtheorem[auto counter, number within=section,
    crefname={Problem}{Problems},
    ]%
    {problem}{Problem}%
    {
        enhanced,
        breakable,
        colback=problemback,
        colframe=primaryblue,
        boxrule=1pt,
        arc=3pt,
        borderline west={3pt}{0pt}{primaryblue},
        fonttitle=\bfseries\color{primaryblue},
        title={Problem \thetcbcounter: #1},
        attach boxed title to top left={yshift=-2mm, xshift=8mm},
        boxed title style={
            boxrule=1pt,
            colback=white,
            colframe=primaryblue,
            arc=2pt
        },
        top=8mm,
        left=4mm,
        right=4mm,
        bottom=4mm,
        drop fuzzy shadow={opacity=0.1}
    }{prob}

%%% ENHANCED SOLUTION ENVIRONMENT
\newtcolorbox{solution}{
    enhanced,
    breakable,
    colback=solutionback,
    colframe=darkgray!30,
    boxrule=0.5pt,
    arc=2pt,
    fonttitle=\bfseries\color{darkgray},
    title=Solution,
    attach boxed title to top left={yshift=-2mm, xshift=8mm},
    boxed title style={
        boxrule=0.5pt,
        colback=white,
        colframe=darkgray!30,
        arc=1pt
    },
    top=8mm,
    left=4mm,
    right=4mm,
    bottom=4mm,
    before skip=10pt,
    after skip=10pt
}

% Note: Simple theorem environments (definition, lemma, etc.) are defined above
% using \newtheorem. For enhanced visual styles, you can replace them with 
% tcolorbox versions if needed.

%%% KEY INSIGHT ENVIRONMENT
\newtcolorbox{keyinsight}{
    enhanced,
    breakable,
    colback=insightback,
    colframe=insightbar,
    boxrule=1pt,
    arc=2pt,
    fonttitle=\bfseries\color{insightbar!80!black},
    title={$\star$ Key Insight},
    attach boxed title to top left={yshift=-2mm, xshift=8mm},
    boxed title style={
        boxrule=1pt,
        colback=white,
        colframe=insightbar,
        arc=1pt
    },
    top=8mm,
    left=4mm,
    right=4mm,
    bottom=4mm,
    before skip=8pt,
    after skip=8pt
}

%%% KEY OBSERVATION ENVIRONMENT  
\newtcolorbox{keyobservation}{
    enhanced,
    breakable,
    colback=lightblue!50,
    colframe=primaryblue,
    boxrule=1pt,
    arc=2pt,
    fonttitle=\bfseries\color{primaryblue},
    title={$\bullet$ Key Observation},
    attach boxed title to top left={yshift=-2mm, xshift=8mm},
    boxed title style={
        boxrule=1pt,
        colback=white,
        colframe=primaryblue,
        arc=1pt
    },
    top=8mm,
    left=4mm,
    right=4mm,
    bottom=4mm,
    before skip=8pt,
    after skip=8pt
}


%------------------------------------------------------------------------------
% 8. CUSTOM MACROS & COMMANDS
%------------------------------------------------------------------------------

% Common mathematical sets
\newcommand{\R}{\mathbb{R}}          % Real numbers
\newcommand{\N}{\mathbb{N}}          % Natural numbers
\newcommand{\Z}{\mathbb{Z}}          % Integers
\newcommand{\Q}{\mathbb{Q}}          % Rational numbers
\newcommand{\C}{\mathbb{C}}          % Complex numbers

% Additional mathematical sets (uncomment as needed)
% \newcommand{\F}{\mathbb{F}}        % Field
% \newcommand{\H}{\mathbb{H}}        % Quaternions
% \newcommand{\P}{\mathbb{P}}        % Projective space

% Delimiters and operators
\DeclarePairedDelimiter{\abs}{\lvert}{\rvert}
\DeclarePairedDelimiter{\norm}{\lVert}{\rVert}
\DeclarePairedDelimiter{\ceil}{\lceil}{\rceil}
\DeclarePairedDelimiter{\floor}{\lfloor}{\rfloor}
\DeclarePairedDelimiter{\inner}{\langle}{\rangle}    % Inner product
\DeclarePairedDelimiter{\bracket}{[}{]}               % Brackets
\DeclarePairedDelimiter{\paren}{(}{)}                 % Parentheses

% Common functions and operators
\newcommand{\indicator}[1]{\mathbf{1}_{#1}}          % Indicator function
\newcommand{\closure}[1]{\overline{#1}}              % Closure
\newcommand{\interior}[1]{\overset{\circ}{#1}}       % Interior
\newcommand{\boundary}[1]{\partial #1}               % Boundary
\newcommand{\powerset}[1]{\mathcal{P}(#1)}          % Power set
\newcommand{\complem}[1]{#1^c}                      % Complement (renamed to avoid conflict)

% Calculus and analysis
\newcommand{\dd}{\mathrm{d}}                          % Differential d
\newcommand{\dv}[2]{\frac{\dd #1}{\dd #2}}          % Derivative
\newcommand{\pdv}[2]{\frac{\partial #1}{\partial #2}} % Partial derivative
% Note: \lim, \limsup, \liminf already defined by LaTeX

% Linear algebra
\newcommand{\rank}{\text{rank}}                      % Rank
\newcommand{\trace}{\text{tr}}                       % Trace
% Note: \det already defined by LaTeX
\newcommand{\spn}{\text{span}}                       % Span (renamed to avoid conflict)
\newcommand{\kernel}{\text{ker}}                     % Kernel
\newcommand{\image}{\text{im}}                       % Image

% Probability and statistics (uncomment as needed)
% \newcommand{\Prob}{\mathbb{P}}                     % Probability
% \newcommand{\Expect}{\mathbb{E}}                   % Expectation
% \newcommand{\Var}{\text{Var}}                      % Variance
% \newcommand{\Cov}{\text{Cov}}                      % Covariance

% Formatting helpers
\newcommand{\important}[1]{\textbf{\color{primaryblue}#1}}
\newcommand{\highlight}[1]{\colorbox{lightblue!50}{#1}}
\newcommand{\marginnote}[1]{\marginpar{\footnotesize\color{darkgray}#1}}
\newcommand{\TODO}[1]{\textcolor{red}{\textbf{TODO: #1}}}
\renewcommand{\qed}{\hfill\ensuremath{\blacksquare}}

% Convenient shortcuts for common phrases
\newcommand{\wolog}{\emph{without loss of generality}}  % Renamed to avoid conflict
\newcommand{\ie}{\emph{i.e.}}
\newcommand{\eg}{\emph{e.g.}}
\newcommand{\cf}{\emph{cf.}}
\newcommand{\etc}{\emph{etc.}}

% Reference shortcuts (use with cleveref)
% Note: \eqref already defined by LaTeX/amsmath
\newcommand{\figref}[1]{Figure~\ref{#1}}
\newcommand{\tabref}[1]{Table~\ref{#1}}


%------------------------------------------------------------------------------
% 9. TITLE METADATA & HEADER/FOOTER
%------------------------------------------------------------------------------
\usepackage{fancyhdr}
\pagestyle{fancy}
\fancyhf{}
\fancyhead[L]{\textsl{\leftmark}}
\fancyhead[R]{\textsl{Solution Manual}}
\fancyfoot[C]{\thepage}
\renewcommand{\headrulewidth}{0.4pt}
\renewcommand{\footrulewidth}{0pt}

% Title page configuration
\title{
    \vspace{-2cm}
    {\Huge\bfseries\color{primaryblue} Solution Manual} \\[0.5cm]
    {\Large\color{darkgray} Course Title or Subject} \\[0.3cm]
    {\large\color{darkgray} Selected Problems and Solutions}
}
\author{
    {\large\bfseries Your Name} \\[0.2cm]
    {\normalsize\color{darkgray} \texttt{your.email@example.com}} \\[0.1cm]
    {\small\color{darkgray} Institution or Affiliation (Optional)}
}
\date{
    {\normalsize\color{darkgray} \today} \\[0.2cm]
    {\small\color{darkgray} Version 1.0}
}

%------------------------------------------------------------------------------
% 10. ADDITIONAL USEFUL ENVIRONMENTS
%------------------------------------------------------------------------------

%%% EXAMPLE ENVIRONMENT
\newtcolorbox{example}{
    enhanced,
    breakable,
    colback=accentgreen!10,
    colframe=accentgreen,
    boxrule=0.5pt,
    arc=2pt,
    fonttitle=\bfseries\color{accentgreen!80!black},
    title=Example,
    attach boxed title to top left={yshift=-2mm, xshift=8mm},
    boxed title style={
        boxrule=0.5pt,
        colback=white,
        colframe=accentgreen,
        arc=1pt
    },
    top=8mm,
    left=4mm,
    right=4mm,
    bottom=4mm,
    before skip=8pt,
    after skip=8pt
}

%%% NOTE/REMARK ENVIRONMENT
\newtcolorbox{note}{
    enhanced,
    breakable,
    colback=warningorange!10,
    colframe=warningorange,
    boxrule=0.5pt,
    arc=2pt,
    fonttitle=\bfseries\color{warningorange!80!black},
    title=Note,
    attach boxed title to top left={yshift=-2mm, xshift=8mm},
    boxed title style={
        boxrule=0.5pt,
        colback=white,
        colframe=warningorange,
        arc=1pt
    },
    top=8mm,
    left=4mm,
    right=4mm,
    bottom=4mm,
    before skip=8pt,
    after skip=8pt
}

%==============================================================================
%                               BEGIN DOCUMENT
%==============================================================================
\begin{document}

% Enhanced Title page
\begin{titlepage}
    \centering
    \vspace*{2cm}
    
    % Title
    {\Huge\bfseries\color{primaryblue} Solution Manual\par}
    \vspace{0.5cm}
    {\Large\color{darkgray} Course Title or Subject\par}
    \vspace{2cm}
    
    % Author info
    \begin{tcolorbox}[
        enhanced,
        colback=white,
        colframe=primaryblue,
        boxrule=1.5pt,
        arc=5pt,
        width=0.7\textwidth,
        center title,
        fonttitle=\large\bfseries,
        title=Author Information
    ]
    \centering
    {\large Your Name}\\[0.3cm]
    {\normalsize your.email@example.com}\\[0.2cm]
    {\small Institution or Affiliation}
    \end{tcolorbox}
    
    \vfill
    
    % Professional description
    \begin{tcolorbox}[
        enhanced,
        colback=lightblue,
        colframe=primaryblue,
        boxrule=1pt,
        arc=3pt,
        width=0.85\textwidth,
        center title,
        fonttitle=\bfseries\color{primaryblue},
        title=About This Solution Manual
    ]
    This manual provides comprehensive solutions to selected problems, featuring:
    \begin{itemize}
        \item Detailed step-by-step solutions
        \item Key insights and observations
        \item Rigorous mathematical proofs
        \item Cross-referenced definitions and theorems
        \item Professional formatting for easy reading
    \end{itemize}
    \end{tcolorbox}
    
    \vfill
    
    % Date and version
    \begin{center}
        {\color{darkgray}\today}\\[0.2cm]
        {\small\color{darkgray}Version 1.0}
    \end{center}
\end{titlepage}

% Professional Preface
\chapter*{Preface}
\addcontentsline{toc}{chapter}{Preface}
\markboth{Preface}{Preface}

\begin{tcolorbox}[
    enhanced,
    colback=lightgray,
    colframe=darkgray,
    boxrule=0.5pt,
    arc=2pt,
    left=4mm,
    right=4mm,
    top=3mm,
    bottom=3mm
]
\textit{Welcome to this comprehensive solution manual. This document is designed 
to provide clear, detailed solutions that enhance understanding of the subject matter.}
\end{tcolorbox}

This solution manual has been carefully prepared to assist students and instructors 
in understanding complex problem-solving techniques. Each solution is presented with:

\begin{itemize}[leftmargin=1.5cm, itemsep=0.3cm]
    \item \textbf{Clear explanations}: Every step is explained in detail
    \item \textbf{Key insights}: Important observations are highlighted
    \item \textbf{Alternative approaches}: Multiple solution methods when applicable
    \item \textbf{Cross-references}: Connections to relevant definitions and theorems
    \item \textbf{Verification}: Solutions include checks where appropriate
\end{itemize}

\section*{How to Use This Manual}

\begin{description}[leftmargin=2cm, itemsep=0.5cm]
    \item[\textbf{Problems}] are presented in blue boxes with clear statements
    \item[\textbf{Solutions}] follow immediately with detailed explanations
    \item[\textbf{Key Insights}] appear in orange boxes highlighting crucial understanding
    \item[\textbf{Key Observations}] are shown in green boxes for important notes
    \item[\textbf{Cross-references}] use the format \texttt{\\cref\{label\}} for easy navigation
\end{description}

\begin{note}
This manual is a living document. If you find errors or have suggestions for 
improvements, please contact the author at the email address provided on the title page.
\end{note}

\section*{Acknowledgments}

Thanks to the mathematical community for providing the foundation upon which these 
solutions are built, and to all students and colleagues who provided feedback during 
the preparation of this manual.

\vfill
\begin{center}
\textit{``Mathematics is not about numbers, equations, computations, or algorithms:}\\
\textit{it is about understanding.''}\\
--- William Paul Thurston
\end{center}

\newpage
\tableofcontents
\newpage

%==============================================================================
%                          TEMPLATE USAGE GUIDE
%==============================================================================
\chapter*{How to Use This Template}
\addcontentsline{toc}{chapter}{How to Use This Template}
\markboth{How to Use This Template}{How to Use This Template}

\begin{note}
This chapter explains how to use the template effectively. You can delete this chapter when creating your actual solution manual.
\end{note}

\section*{Available Environments}

\subsection*{Problem and Solution Environments}

The template provides enhanced environments for problems and solutions:

\textbf{Basic Usage:}
\begin{verbatim}
\begin{problem}{Problem Title}{label:unique-identifier}
    State your problem here.
\end{problem}

\begin{solution}
    Provide your solution here.
\end{solution}
\end{verbatim}

\subsection*{Enhanced Solution Features}

\textbf{Key Insight Environment:}
\begin{verbatim}
\begin{keyinsight}
    The crucial insight is that...
\end{keyinsight}
\end{verbatim}

\begin{keyinsight}
    Example: This insight box helps highlight the most important ideas in your solution.
\end{keyinsight}

\textbf{Key Observation Environment:}
\begin{verbatim}
\begin{keyobservation}
    Notice that this approach works because...
\end{keyobservation}
\end{verbatim}

\begin{keyobservation}
    Example: Use observations to point out important details that students might miss.
\end{keyobservation}

\subsection*{Mathematical Environments}

\textbf{Definitions:}
\begin{verbatim}
\begin{definition}[Optional Name]
    A formal definition goes here.
\end{definition}
\end{verbatim}

\begin{definition}[Limit]
    We say that $\lim_{x \to a} f(x) = L$ if for every $\varepsilon > 0$, there exists $\delta > 0$ such that whenever $0 < |x - a| < \delta$, we have $|f(x) - L| < \varepsilon$.
\end{definition}

\textbf{Lemmas and Theorems:}
\begin{verbatim}
\begin{lemma}[Optional Name]
    Statement of the lemma.
\end{lemma}

\begin{theorem}[Optional Name]
    Statement of the theorem.
\end{theorem}
\end{verbatim}

\section*{Mathematical Typesetting Tips}

\subsection*{Built-in Mathematical Shortcuts}

The template includes many useful shortcuts:

\begin{itemize}
    \item \texttt{\\dd} for differential d: $\int f(x) \dd x$
    \item \texttt{\\dv\{f\}\{x\}} for derivatives: $\dv{f}{x}$
    \item \texttt{\\pdv\{f\}\{x\}} for partial derivatives: $\pdv{f}{x}$
    \item \texttt{\\complem\{A\}} for complement: $\complem{A}$ (renamed to avoid conflicts)
    \item \texttt{\\closure\{A\}} for closure: $\closure{A}$
    \item \texttt{\\interior\{A\}} for interior: $\interior{A}$
    \item \texttt{\\boundary\{A\}} for boundary: $\boundary{A}$
\end{itemize}

\subsection*{Formatting Helpers}

\begin{itemize}
    \item \texttt{\\important\{text\}} for \important{important text}
    \item \texttt{\\highlight\{text\}} for \highlight{highlighted text}
    \item \texttt{\\wolog} for \wolog
    \item \texttt{\\ie} for \ie, \texttt{\\eg} for \eg
\end{itemize}

\section*{Cross-Referencing}

Use meaningful labels and the \texttt{cleveref} package for smart references:

\begin{verbatim}
\begin{problem}{Sample Problem}{prob:sample}
    This is a sample problem.
\end{problem}

Later, reference it with: \cref{prob:sample}
\end{verbatim}

The template supports:
\begin{itemize}
    \item \texttt{\\cref\{label\}} for smart references
    \item \texttt{\\figref\{label\}} for figures
    \item \texttt{\\tabref\{label\}} for tables
    \item Standard \texttt{\\eqref\{label\}} for equations
\end{itemize}

\section*{Example and Note Environments}

\textbf{Examples:}
\begin{verbatim}
\begin{example}
    Show that the function f(x) = x² is continuous.
\end{example}
\end{verbatim}

\begin{example}
    To show continuity of $f(x) = x^2$ at $x = a$, we compute:
    $$\lim_{x \to a} f(x) = \lim_{x \to a} x^2 = a^2 = f(a)$$
\end{example}

\textbf{Notes:}
\begin{verbatim}
\begin{note}
    Additional information or remarks.
\end{note}
\end{verbatim}

\section*{Color Customization}

To customize colors, modify these definitions in the preamble:

\begin{verbatim}
\definecolor{primaryblue}{RGB}{41, 98, 255}      % Main accent
\definecolor{problemback}{RGB}{232, 240, 254}    % Problem background
\definecolor{solutionback}{RGB}{249, 250, 251}   % Solution background
\end{verbatim}

\section*{Compilation Instructions}

\begin{note}
For best results, compile the document as follows:

\begin{enumerate}
    \item \texttt{pdflatex filename.tex} (first pass)
    \item \texttt{pdflatex filename.tex} (second pass for cross-references)
    \item \texttt{makeindex filename.idx} (if using index)
    \item \texttt{pdflatex filename.tex} (final pass)
\end{enumerate}

Most modern LaTeX editors handle this automatically.
\end{note}

\section*{Getting Started}

\begin{enumerate}
    \item \textbf{Customize the title page}: Update the author name, course title, email, and institution in the preamble.
    \item \textbf{Modify chapters and sections}: Replace the placeholder chapter and section titles with your content.
    \item \textbf{Add your content}: Replace the example problems and solutions with your actual material.
    \item \textbf{Compile the document}: Run \texttt{pdflatex} multiple times to ensure proper cross-references.
\end{enumerate}

\section*{Available Environments}

This template provides several custom environments:

\begin{description}
    \item[\texttt{problem}] For stating problems with automatic numbering
    \item[\texttt{solution}] For providing detailed solutions
    \item[\texttt{keyinsight}] For highlighting crucial insights within solutions
    \item[\texttt{keyobservation}] For noting important observations
    \item[\texttt{example}] For providing illustrative examples
    \item[\texttt{note}] For important remarks or warnings
\end{description}

\section*{Mathematical Commands}

The template includes many predefined mathematical commands. See the preamble for a complete list, including:

\begin{itemize}
    \item Common sets: \verb|\R|, \verb|\N|, \verb|\Z|, \verb|\Q|, \verb|\C|
    \item Delimiters: \verb|\abs{x}|, \verb|\norm{x}|, \verb|\inner{x,y}|
    \item Functions: \verb|\indicator{A}|, \verb|\closure{A}|, \verb|\interior{A}|
    \item Calculus: \verb|\dv{f}{x}|, \verb|\pdv{f}{x}|
\end{itemize}

\begin{example}
Here's how to use some mathematical commands:
\begin{align}
    f: \R \to \R, \quad f(x) = \abs{x}^2 \\
    \norm{x + y} \leq \norm{x} + \norm{y} \quad \text{(Triangle inequality)}
\end{align}
\end{example}

\newpage

%==============================================================================
%                             ACTUAL CONTENT
%==============================================================================

\chapter{Chapter Title 1}
\thispagestyle{fancy}

% Chapter introduction
\begin{tcolorbox}[
    enhanced,
    colback=lightgray,
    colframe=darkgray,
    boxrule=0.5pt,
    arc=2pt,
    left=4mm,
    right=4mm,
    top=3mm,
    bottom=3mm
]
\textit{This chapter covers fundamental concepts in your subject area. 
Add a brief description of the topics covered in this chapter.}
\end{tcolorbox}

\section{Section Title 1}

\begin{definition}
A \textbf{special property} is a mathematical concept that satisfies the following conditions:
\begin{enumerate}
    \item It preserves the fundamental structure of the object.
    \item It is invariant under transformations.
    \item It can be characterized in multiple equivalent ways.
\end{enumerate}
\end{definition}

\begin{lemma}
Every object with the special property has a unique canonical representation.
\end{lemma}
\begin{proof}
The proof follows from the definition and basic properties of the underlying structure. We proceed by construction...
\end{proof}

\begin{problem}{This is Problem 1}{problem1}
    This is problem 1. Write your problem statement here.
\end{problem}
\begin{solution}
    Solution to problem 1 goes here.
\end{solution}

\begin{problem}{This is Problem 2}{problem2}
    This is problem 2. Write your problem statement here.
\end{problem}
\begin{solution}
    Solution to problem 2 goes here.
\end{solution}

\begin{problem}{This is Problem 3}{problem3}
    This is problem 3. Write your problem statement here.
\end{problem}
\begin{solution}
    Solution to problem 3 goes here.
\end{solution}

\chapter{Chapter Title 2}
\thispagestyle{fancy}

% Chapter introduction
\begin{tcolorbox}[
    enhanced,
    colback=lightgray,
    colframe=darkgray,
    boxrule=0.5pt,
    arc=2pt,
    left=4mm,
    right=4mm,
    top=3mm,
    bottom=3mm
]
\textit{This chapter explores fundamental concepts in your subject area. 
Add a brief description of the topics covered in this chapter.}
\end{tcolorbox}

\section{Section Title 2}

\begin{problem}{This is Problem 4}{problem4}
    This is problem 4. Write your problem statement here.
\end{problem}
\begin{solution}
    Solution to problem 4 goes here.
\end{solution}

\begin{problem}{This is Problem 5}{problem5}
    This is problem 5. Write your problem statement here.
\end{problem}
\begin{solution}
    Solution to problem 5 goes here.
    
    \begin{keyobservation}
        The key observation is that this problem can be decomposed into two simpler subproblems, each of which can be solved independently.
    \end{keyobservation}
    
    Subproblem 1: We first show that...
    
    Subproblem 2: Next, we establish that...
    
    \begin{keyinsight}
        The crucial insight is that combining the results from both subproblems yields a contradiction unless the desired conclusion holds.
    \end{keyinsight}
    
    Therefore, the proof is complete.
\end{solution}

\end{document}